\chapter{Experimental Setup} % (fold)
\label{sec:experimental_setup}
\todo[inline]{Write Setup of experiment. Experiments on classification (caltech, VOC) first to show the workings (NBNN with fg/bg, with and without behmo), and then on detection. Explain the evaluation method of VOC, explain to what the results are compared}

\todo[inline]{Make a list of experiments}
\section{NBNN-cls} % (fold)
\label{sub:nbnn-cls}

% subsection nbnn-cls (end)

\section{NBNN detection} % (fold)
\label{sub:nbnn_detection}

% subsection nbnn_detection (end)

\section{Influence of clustering algorithms} % (fold)
\label{sub:influence_of_clustering_algorithms}

% subsection influence_of_clustering_algorithms (end)

\section{local NBNN detection} % (fold)
\label{sub:local_nbnn_detection}
\todo[inline]{k>1, McCann}

% subsection local_nbnn_detection (end)

\section{Training weighted distances} % (fold)
\label{sub:training_weighted_distances}
\todo[inline]{Put Behmo here}
% subsection training_weighted_distances (end)

\section{Detection} % (fold)
\label{sub:detection}
NBNN-based-detection is tested on the VOC2012 \todo{and VOC2007? misschien goed voor vergelijkingen, if there's time...} data set. The performance of detection is measured both with and without optimization of the class-\-specific pa\-ra\-me\-ters defined by \cite{behmo2010towards}. A part of the training set was set apart as a optimization set, the rest was used as labeled examples. The full validation set was used for measuring performance.

SIFT descriptors were densely sampled with a spacing of 6\todo{3 would be ideal, try if that might work)} pixels and on 4 scales (1.33, 2.0, 3.0 and 4.5 \todo{what? sigma, lookup}) from all example images, just like \cite{mccann2011local} proposes in their setup. 


Detection of each object class is modeled as a separate 2-class foreground-background detection problem. All images containing one or more objects of an object class are selected, and for each image, the descriptors inside the class's bounding boxes are added to the foreground-class, while the descriptors outside these boxes are added to the background-class.

For all foreground descriptors an exemplar is stored, containing the relative location of the object within its bounding box and the relative size of the bounding box compared with the descriptor's scale.

When oNBNN is used, the optimization set is used to do a 
\todo{tell about behmo}
\todo[inline]{tell about testing -> first NN, then get exemplars, which makes hypotheses. Then find pairwise overlap between hypotheses, then cluster overlap, then take out largest cluster and make a detection of it, remove overlaps with this cluster, repeat clustering until no hypothesis left, ranking of detections (Qd, Qh)}
% subsection detection (end)

\section{VOC data set} % (fold)

\label{sub:voc_data_set}
The Visual Object Classes (VOC) challenge of the PASCAL network \todo[fancyline]{reference} provides a popular data set annotated for image detection\todo[fancyline]{reference}, and will form the main task to perform. Because it is popular, a lot of comparison with other methods is possible, among which some state of the art methods and similar approaches.

\todo[inline]{Needs more elaboration, or not really?}
% subsection voc_data_set (end)

\section{TUDmotorbikes data set} % (fold)
\label{sub:tudmotorbikes_data_set}

% subsection tudmotorbikes_data_set (end)

% section experimental_setup (end)