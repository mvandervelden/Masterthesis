
\section{Conclusion} % (fold)
\label{cha:conclusion}
\todo[inline]{Write Conclusion. compared to each other and others. Relate back to the theory, and explain why predictions have or have not come true}

Both in theory and in performance, NBNN seems to be a good idea, not only on the image classification task, but also on object detection, as I have shown in this thesis. By combining an existing approach for NBNN object detection using exemplars with the theoretically sound concepts of quickshift clustering and LNBNN classification, an approach was created that is capable of achieving good results on a popular benchmark image set.

The main disadvantage of the method used is its complexity in both time and memory. While training is fairly easy and fast, the actual detection phase proves to be the main bottleneck. Mostly because of this, LNBNN-detection would not be a preferred method to use at the time. There might however be some room for improvement.

\todo{TODO}

\subsection{Future Work} % (fold)
\label{sec:future_work}


\todo[inline]{Shortly talk about improving on time/memory consumption, efficient clustering algorithms, the need to perform pairwise comparison on all hypotheses.}

% section future_work (end)

% chapter conclusion (end)